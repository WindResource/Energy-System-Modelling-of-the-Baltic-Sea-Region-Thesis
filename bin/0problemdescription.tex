\section{Problem Description}

This technical report outlines the steady-state Reynolds-Averaged Navier-Stokes (RANS) simulations conducted on the NLR7301 two-element airfoil, comparing a coarse and fine mesh resolution, 6 and 13.1 degrees angle of attack, and different modelling schemes and turbulence models. Additionally, the results are compared with experimental references. \\

The flow conditions are characterised by a chord-length Reynolds number of 2.51 million and a Mach number of 0.185. The ambient flow conditions are considered with a static temperature and pressure of 293 K and 101325 Pa. The fluid is modelled as a compressible ideal gas with a dynamic viscosity according to Sutherland's law and a turbulence intensity of 1\%. \\

The simulations are carried out at two distinct angles of attack, namely 6 and 13.1 degrees, operating under specific flow conditions characterised by a chord-length Reynolds number of 2.51 million and a Mach number of 0.185. The wing and flap coordinates are normalised relative to the reference chord length based on the wing chord with the flap retracted.