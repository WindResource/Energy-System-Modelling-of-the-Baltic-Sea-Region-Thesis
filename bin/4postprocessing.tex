\section{Post-Processing}
This section discusses the results generated by the CFX-solver manager and processed by CFX-Post, a post-processing tool in ANSYS CFX used for visualizing and analyzing computational fluid dynamics (CFD) simulation data.\\

To quantify the effect of numerical errors on the results, the simulation of the two-element airfoil with the fine mesh and a 13.1-degree angle of attack is repeated with a first-order upwind scheme instead of a high-resolution scheme defined in CFX-pre. The first-order upwind scheme is a simple and computationally less expensive method that tends to be less accurate near discontinuities than the high-resolution scheme.\\

To quantify the effect of the turbulence model on the results, the simulation of the two-element airfoil with the fine mesh and a 13.1-degree angle of attack is repeated with a shear stress transport (SST) turbulence model instead of a generalized k-omega turbulence model, as defined in CFX-pre. The SST turbulence model is known for handling complex flow situations near walls and separation regions very well, making it more accurate than the generalized k-omega turbulence model.





\subsection{Wall Coordinate}
The wall coordinate y+ is a dimensionless parameter characterising the mesh resolution near solid boundaries, such as walls. It represents the distance from a fluid cell's centre to the nearest wall, normalised by the viscous length scale, typically the kinematic viscosity and the friction velocity. \\

For y+ values less than 1, the mesh resolution near the wall is sufficiently fine to resolve the viscous sublayer and provide accurate predictions of boundary layer behaviour. For values greater than 30, the mesh resolution is relatively coarse, and the flow near the wall is not adequately resolved, leading to inaccuracies in predicting the boundary layer properties. Therefore, choosing an appropriate resolution for the wall coordinate y+ is essential for accurately modelling the boundary layer flow. \\

Figure \ref{fig:wallcoordinate} illustrates the wall coordinate y+ as a function of the X/C coordinate of the wing and flap element boundaries, modelled in a fine and coarse mesh. The X/C coordinate is the airfoil coordinate, normalised by its reference chord length. The wall coordinate y+ is a solution variable found in CFX-post. The figure shows that y+ significantly reduces for all coordinates of the wing and flap elements when choosing a finer mesh, which is to be expected. \\

The maximum value of y+ for the fine mesh, at around 60, is significantly lower than the corresponding value for the coarse mesh, at around 200. A lower value of y+ predicts a higher accuracy for modelling the boundary layer flow. Finer mesh resolutions have been attempted to achieve the maximum y+ needed to resolve the viscous boundary layer. However, during the simulation, persistent fluctuations in the RMS turbulence prevented convergence in the CFX-Solver Manager. This lack of convergence is likely caused by inadequate mesh quality resulting from the mesh refinement.



\begin{figure}[H]
  \centering
    \includegraphics[width=0.7\textwidth]{Fig/6degrees_Yplus.png}
  \caption{The wall coordinate y+ as a function of the X/C coordinate for the wing and flap element boundaries for a coarse and fine mesh.}
  \label{fig:wallcoordinate}
\end{figure}


\subsection{Pressure and Velocity}
Figure \ref{fig:contour} depicts contour figures of the static pressure and velocity around the two-element airfoil at a 6 and 13.1-degree angle of attack, comparing the contours. Analysing the static pressure figures \ref{fig:contour6P} and \ref{fig:contour13.1P}, it is evident that increasing the angle of attack from 6 to 13.1 degrees results in higher pressure gradients on the airfoil surface. The more pronounced red region on the upper surface in Figure \ref{fig:contour13.1P} indicates the higher pressure gradient, suggesting a stronger low-pressure area characteristic of increased lift. However, this also indicates the potential for flow separation, as the adverse pressure gradient is relatively strong.\\


For the velocity plots \ref{fig:contour6V} and \ref{fig:contour13.1V}, the 6-degree angle of attack displays a smoother velocity distribution. A blue region near the leading edge indicates low velocity, transitioning into green and yellow for higher velocity along the surface. In contrast, the 13.1-degree displays a high velocity with a red region closer to the leading edge on the upper surface. This region indicates accelerated flow due to the increased angle of attack and the beginning of flow separation as the flow approaches the trailing edge.



\begin{figure}[H]
  \centering
  % Third pair of coarse and fine mesh images
  \begin{subfigure}[b]{0.49\textwidth}
    \includegraphics[width=\textwidth]{Fig/6degrees_fine_contour_P_new.png}
    \caption{Static pressure, 6-degree AoA}
    \label{fig:contour6P}
  \end{subfigure}
  \hfill
  \begin{subfigure}[b]{0.49\textwidth}
    \includegraphics[width=\textwidth]{Fig/13.1degrees_fine_contour_P_new.png}
    \caption{Static pressure, 13.1-degree AoA}
    \label{fig:contour13.1P}
  \end{subfigure}

  % Fourth pair of coarse and fine mesh images
  \begin{subfigure}[b]{0.49\textwidth}
    \includegraphics[width=\textwidth]{Fig/6degrees_fine_contour_V_new.png}
    \caption{Velocity, 6-degree AoA}
    \label{fig:contour6V}
  \end{subfigure}
  \hfill
  \begin{subfigure}[b]{0.49\textwidth}
    \includegraphics[width=\textwidth]{Fig/13.1degrees_fine_contour_V_new.png}
    \caption{Velocity, 13.1-degree AoA}
    \label{fig:contour13.1V}
  \end{subfigure}

  \caption{Contour figures of the static pressure and velocity around the two-element airfoil at a 6 and 13.1-degree angle of attack (AoA) for a fine mesh generated in CFX-Post.}
  \label{fig:contour}
\end{figure}


\subsection{Pressure Coefficient}
Figure \ref{fig:Cp} presents the pressure coefficients around the wing and flap elements of the two-element airfoil for the modelled compared to the experimental reference data at a 6 and 13.1-degree angle of attack as a function of the X/C coordinate. \\

The coarse and fine mesh simulations align well with the experimental data across the wing element's chord (Figure \ref{fig:Cp6wing}) for the 6-degree angle of attack, suggesting that mesh refinement minimally impacts the simulated pressure distribution. However, discrepancies near the leading edge highlight potential limitations of the simulation models or experimental differences. Using both coarse and fine meshes, the modelled data for the flap element generally follows the experimental trend (Figure \ref{fig:Cp6flap}), with slight deviations at the leading edge indicating an overestimated pressure. The models align well with the experimental data near the trailing edge, showing accurate modelling of the pressure recovery.\\

The fine mesh with the shear stress transport (SST) model shows a close approximation to the experimental data along the chord of the wing element at a 13.1-degree angle of attack (Figure \ref{fig:Cp13.1wing}), suggesting it captures the flow physics accurately. The First Order Upwind scheme shows significant deviation, especially in areas with steep pressure gradients. This deviation suggests that it is less effective in capturing complex flow features. The SST model's results are closer to the experimental data for the flap element (Figure \ref{fig:Cp13.1flap}), indicating a more accurate representation of the pressure distribution. The first-order upwind scheme diverges more from the experimental curve, especially towards the trailing edge, suggesting less precision in modelling the complex flow features present in this region.\\


The pressure coefficient ($C_p$) around the wing and flap elements of the two-element airfoil is calculated by setting up an expression around the boundaries in CFX-Post. Since the free-stream pressure ($p_{\infty}$) is assumed to be the relative value of the static pressure, 0 Pa, the equation is:

\begin{equation}
C_p = \frac{p}{\frac{1}{2} \rho_{\infty} V_{\infty}^2}
\end{equation}

$p$ is the local pressure, $\rho_{\infty}$ is the free-stream density of the air, 1.185 ($kg m^-3$), and $V_{\infty}$ is the free-stream velocity. \\

The free-stream velocity ($V_{\infty}$) is calculated based on the given Mach number and the speed of sound in an ideal gas. The Mach number is the ratio of the object's speed to the speed of sound in the medium. Given the Mach number of 0.185, the free stream velocity is calculated with the equation:

\begin{equation}
V_{\infty} = M \cdot a = M \cdot \sqrt{\gamma R T}
\end{equation}

In this case, M is the Mach number, a is the speed of sound, $\gamma$ is the heat capacity ratio, 1.4 for air, $R$ represents the specific gas constant, 287 $J/kg \cdot K$ for air, and $T$ is the static temperature, 293 $K$.

\subsection{Lift and Drag Coefficient}
Table \ref{tab:Cliftdrag} presents the lift and drag coefficients determined by the solution variable of CFX-post. Figure \ref{fig:Cliftdrag} compares the drag coefficients to the reference experimental values. This comparison is obtained by calculating the difference between the modelled value and the reference value and normalising that value by the reference value. \\

Figure \ref{fig:Cliftdrag} indicates that for a two-element airfoil, changes in mesh density have minimal impact at a 6-degree angle of attack. At a 13.1-degree angle, however, the choice of numerical scheme and turbulence model greatly influences the results. The first-order upwind scheme shows a notable underprediction of drag, highlighting its inaccuracy in complex flow conditions. The SST turbulence model, in contrast, yields better lift and drag predictions than the k-omega model, demonstrating its effectiveness in resolving intricate flow features at higher angles of attack. These observations suggest that investment in a more sophisticated turbulence model and a higher-order numerical scheme is justified for simulations where accuracy in complex flow regions is crucial.


\begin{figure}[H]
  \centering
  % Third pair of coarse and fine mesh images
  \begin{subfigure}[b]{0.6\textwidth}
    \includegraphics[width=\textwidth]{Fig/6degrees_Cp_wing.png}
    \caption{Wing element, 6-degrees AoA}
    \label{fig:Cp6wing}
  \end{subfigure}
  \hfill
  \begin{subfigure}[b]{0.6\textwidth}
    \includegraphics[width=\textwidth]{Fig/6degrees_Cp_flap.png}
    \caption{Flap element, 6-degrees AoA}
    \label{fig:Cp6flap}
  \end{subfigure}

  % Fourth pair of coarse and fine mesh images
  \begin{subfigure}[b]{0.6\textwidth}
    \includegraphics[width=\textwidth]{Fig/13.1degrees_Cp_wing.png}
    \caption{Wing element, 13.1-degrees AoA}
    \label{fig:Cp13.1wing}
  \end{subfigure}
  \hfill
  \begin{subfigure}[b]{0.6\textwidth}
    \includegraphics[width=\textwidth]{Fig/13.1degrees_Cp_flap.png}
    \caption{Flap element, 13.1-degrees AoA}
    \label{fig:Cp13.1flap}
  \end{subfigure}

  \caption{The pressure coefficient as a function of the X/C coordinate comparing modelled data to the reference experimental data of the wing and flap elements of the two-element airfoil for 6 and 13.1-degree angle of attacks (AoAs).}
  \label{fig:Cp}
\end{figure}





\begin{table}[H]
\centering
\caption{Lift and drag coefficients of the modelled values compared to the reference values.}
\label{tab:Cliftdrag}
\resizebox{0.75\columnwidth}{!}{%
\begin{tabular}{llll}
\textbf{Angle of Attack} & \textbf{Characteristic} & \textbf{Lift Coefficient ($C_L$)} & \textbf{Drag Coefficient ($C_D$)} \\ \hline
\multicolumn{1}{l|}{6 Degrees}    & \multicolumn{1}{l|}{Coarse Mesh}              & 2.476          & 0.0225          \\
\multicolumn{1}{l|}{}             & \multicolumn{1}{l|}{Fine Mesh}                & 2.485          & 0.0234          \\
\multicolumn{1}{l|}{}             & \multicolumn{1}{l|}{\textbf{Reference Value}} & \textbf{2.416} & \textbf{0.0229} \\ \hline
\multicolumn{1}{l|}{13.1 Degrees} & \multicolumn{1}{l|}{Fine Mesh}                & 3.085          & 0.0654          \\
\multicolumn{1}{l|}{}             & \multicolumn{1}{l|}{First Order Upwind}       & 2.196          & 0.0254          \\
\multicolumn{1}{l|}{}             & \multicolumn{1}{l|}{Shear Stress Transport}   & 3.147          & 0.0670          \\
\multicolumn{1}{l|}{}             & \multicolumn{1}{l|}{\textbf{Reference Value}} & \textbf{3.141} & \textbf{0.0445}
\end{tabular}%
}
\end{table}


\begin{figure}[H]
  \centering
    \includegraphics[width=0.5\textwidth]{Fig/Normalised_Lift_Drag.png}
  \caption{Lift and drag coefficients of the modelled values compared and normalised to the reference values.}
  \label{fig:Cliftdrag}
\end{figure}
